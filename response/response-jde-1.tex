\documentclass[letterpaper,12pt]{article}

% Xetex preamble
\usepackage{fontspec}
\setromanfont[Ligatures=TeX]{TeX Gyre Pagella}
\usepackage{unicode-math}
\setmathfont{TeX Gyre Pagella Math}

\usepackage[longnamesfirst]{natbib}
%\usepackage[tabhead,nolists,tablesfirst]{endfloat}
\usepackage[flushleft]{threeparttable}
\usepackage{booktabs}
\usepackage{rotating}  
\usepackage{dcolumn} 
\usepackage{setspace}
%\usepackage{flafter} 
%\usepackage{longtable}
%\usepackage[pdftex]{graphicx}
\usepackage[xetex,colorlinks=true,linkcolor=black,citecolor=black]{hyperref} 
\usepackage[margin=1.0in]{geometry} 
\usepackage{multirow}

%opening
\title{} \author{}

\doublespacing

\begin{document}

\begin{center} \textbf{\large Income Shocks, Contraceptive Use, and
Timing of Fertility} \end{center}

\begin{center} Response to Referee Comments \end{center}

\noindent Please find attached the revised version of our paper,
``Income Shocks, Contraceptive Use, and Timing of Fertility.''
We would like to thank the two referees for the very useful comments and
suggestions,
which we believe have improved the paper.

We rewrote the code for cleaning and analyzing the data from the ground up 
to make it easier to follow and eventually publish on GitHub.
This rewrite lowered the sample by three observations.
Including all married/partnered women up to age 45, rather than exclude
women younger than 18, the final sample size is, however, the same as 
before and the results are essentially unchanged.

To remain within the page limit we had to substantially reorganize
part of the paper.
The following is a list of the more substantial changes made:
\begin{enumerate}
\item We removed the map of Kagera.
\item In response to the reviewers' comments we have almost completely
rewritten the Robustness section as detailed below.
\item As part of the rewrite of the Robustness section we moved 
the discussion of age, wealth, and education results to a new Appendix
section since these results do little to clarify the mechanisms 
behind the main results.
We still address and incorporate the comments from reviewers that
relate to these outcomes.
\item We moved the discussion of estimation strategy for age,
wealth, and education estimation to the new Appendix section.
\item We have added another new Appendix section that examine results 
with including an additional lag for crop loss (see response to
Reviewer 1 below).
\item A number of new Appendix tables have been added in response
to various comments.
\end{enumerate}

We have tried to highlight the factors that can affect our identification 
strategy and reduce the use of causal language, but if we need to tone down 
the causality claims further we can do so.
We list below our responses to the individual comments and suggestions.



\newpage

\section*{Reviewer 1}

% This interesting paper analyzes the relationship between income shocks
% (as proxied by self-reported crop loss) and marital fertility, using a
% small Tanzanian panel survey from the early 1990s. A key feature of the
% paper is that the authors are able to observe use of both traditional
% and modern birth control techniques. In regression specifications with
% and without community or woman fixed effects, the authors find that crop
% loss is associated with decreases in the probability of conceiving, with
% a corresponding increase in use of traditional birth control techniques
% but not modern contraceptives.

The paper could benefit from focusing its motivation somewhat more. An
important point (that the authors do not emphasize enough) is that use
of modern contraception in the sample is extremely low, so the lack of a
relationship between shocks and use of modern methods is not surprising.
The more noteworthy result is that use of traditional methods responds
to shocks. In this sense, the paper makes an important contribution to
understanding how populations control fertility before widespread
adoption of modern contraception. For instance, it is well known that
marital fertility in Europe declined largely before modern contraception
became available, but the methods couples used to achieve this reduction
are not well documented (see, for example, Guinnane JEL 2011). The paper
would be even more compelling if the authors focused the paper's
motivation on the questions of whether and how couples modify fertility
to respond to changing economic conditions before the advent of
modern contraception. That motivation would also help justify their
choice of dataset, which otherwise seems somewhat small and out of date
relative to other available panel datasets from developing countries.

[RESPONSE:]

We have edited the Introduction to highlight the role of traditional 
contraception and that most of the increase in contraception after a 
shock comes from traditional methods.
We have also added an additional point to our list of the main 
contributions of the paper reflecting your suggestion.
In addition to the \citet{Guinnane2011} paper, we also cite a just
published paper on spacing in pre-fertility transition England
\citep{Cinnirella2017} and \citet{Michael1976}, \citet{David1986},
and \citet{Santow1995}
on the role of different types of traditional contraception in
controlling fertility before the emergence of modern contraception.

In the Conclusion we already discussed that our results were
especially of interest because of the composition of contraceptive
methods used.
We have edited the discussion to further clarify the major role
of traditional contraceptives.
In addition, we have recast the second to last paragraph to
highlight the connection to long-term fertility and the literature
on fertility transition in Europe and the US.

% [should the footnote on historical control of fertility be moved up
% to why this is important?]

% [few data sets with panel data on contraceptive use and economic
% circumstances]

\noindent \textbf{Specific comments:}

\begin{enumerate}

\item The authors interpret their results as postponement, implying they
represent a change in the timing of fertility but not its lifetime
cumulation. That is likely to be the correct interpretation, but can
they really tell? It seems to me that the current fixed effect model
does not tell us. Perhaps they could include 1-2 lags in the crop shock
variable to shed some light on it.

[RESPONSE:]

It is correct that the current fixed effects model cannot tell
us whether we observe postponement of fertility or a permanent reduction.
We focus on postponement for two reasons.

First, we consider the ability to actively postpone fertility as a 
necessary---but not sufficient---condition for permanent fertility reduction.
In other words, we see postponement as a weaker claim than permanent 
fertility reduction.
If we cannot show a conscious decision to delay fertility under the
right circumstances, it is hard to argue that we can convincingly
show find evidence of conscious, permanent reduction in fertility.
Note that this does not follow the original demography
literature, but is in line with the more recent research
\citep[See discussion in][]{Cinnirella2017}.

Second, from a theoretical standpoint it is not clear that 
shocks---in and off themselves---should substantially affect completed 
fertility. 
We now discuss evidence from Guatemala that the overall \emph{risk} of 
experiencing a shock is likely to affect fertility, but that only
shocks that occur late in a woman's reproductive life cycle is
likely to affect completed fertility (because the opportunities to
make up for the ``missed'' birth are fewer and because the marginal
benefit of a child is smaller the more children you already have and
the older you are).
Hence, an important distinction is that long term fertility is likely
affected by the overall \emph{risk}, whereas the exact timing of
fertility is driven by a combination of risk and shocks, but probably
mostly shocks.

% The longitudinal nature of the data is important for our ability
% to control for unobservable village, household, and individual 
% characteristics, but the trade-off is that the data only cover 
% 32 months and that is too short to analyze long-term fertility
% outcomes.
That said, we did follow your suggestion of adding an additional 
lag to the model and discuss the results in a new Appendix section.
In principle, if shocks lead to a postponement of fertility,
crop losses that occurred further in the past should be
associated with an \emph{increased} likelihood of current pregnancy 
or birth (an initial reduction in fertility should be followed
by a compensatory increase later on), 
whereas if shocks lead to a permanent reduction in fertility we
should not see any compensatory effect.

Although the results are potentially interesting in themselves,
because of data limitation the approach unfortunately does not  
provide a clean test for whether households postpone or reduce their 
fertility in response to shocks.
First, given the short duration between surveys and the lack of 
information about when the shocks occurred, it is possible and
even likely that the effects of a crop loss last for more than one period.
Hence, if we find a negative effect of more distant crop losses
this may be the result of the negative effects of the crop losses
still affecting the household.
Second, we lose 1/4 of the original number of observations each 
time we include an additional lag, which leaves us with less variation 
in crop loss because the first round had a higher proportion of crop 
losses than the subsequent rounds.

Even if crop losses' impact are limited to only one period, there 
are two additional concerns.
First, the compensatory effect may be spread over multiple 
periods because of the uncertainty associated with waiting time
to conception, i.e. a couple may be trying to conceive in the
period after the crop loss but fail.
This makes a positive compensatory effect harder to 
identify.
Second, it is possible that households may respond differently to a 
shock if they are hit repeatedly.
There are two competing factors at work.
On one hand, repeated shocks may exacerbate the negative impact on the
household, making it \emph{more} likely that fertility is postponed.
On the other hand, even longer spacing between births can be costly 
because economies of scale are harder to realize and because the 
mother may have to  be out of the labor force for longer if she 
already have young children to care for, which will make it
\emph{less} likely that fertility is postponed.
To capture the idea that repeated shocks have an additional
impact we therefore also include the interaction between 
crop losses. 

With the data limitations and additional concerns,
it is difficult to draw strong predictions on what the coefficients
should be (except that the initial impact should lead to postponement 
of fertility).
Including only the additional lag of crop loss it appears that 
there is a weak positive association between crop loss in the prior
period and the likelihood of pregnancy and birth, although the 
coefficients are small and statistically insignificant.
There is still a negative effect of the immediate crop loss
variable, although these are also not statistically significant.
These results are broadly consistent with a compensatory response
where crop loss longer ago leads to higher fertility now.
However, if we include the interaction between crop losses we 
instead find negative association for both immediate and lagged 
crop loss with a positive interaction coefficient.
Even with the positive interaction coefficient the estimated 
impact of crop losses in the two prior periods is negative.
For contraceptive use we see similar patterns (with the signs 
reversed).


Although these results should be interpreted with caution, they
are not inconsistent with a compensatory/postponement effect, 
as long as some crop losses' impact last for more than one
period after its occurrence and/or the compensatory effect is
spread over more than one period.
In either specification the coefficients on lagged crop loss
are small relative to the immediate impact.
In sum, we cannot establish with any degree of confidence from
these results whether the response to shocks lead to only postponement 
of shocks or a permanent reduction.
Given the findings from Guatemala that shocks only 
have a negative impact on completed fertility if they occur
late in a woman's reproductive life and that completed 
fertility is mainly driven by the \emph{risk} of shocks,
it is most likely that what we observe are, indeed, postponement
rather than a reduction in completed fertility \citep{Portner2014}.

\item The authors argue that the increase in modern contraceptive use is
intentional, but the evidence for this interpretation is weaker than
they would like. As they do note, while the survey asks about using
abstinence specifically to prevent births, there is much scope for
misreporting, and many women may claim that reductions in coital
frequency were related to fertility planning even if they were not. As
their main evidence against this concern, the authors show that crop
loss has small effects on labor supply. But this finding is only
relevant for the very specific theory that all unintentional abstinence
is related to a work-sex tradeoff. Couples may have less sex during hard
times because they are stressed, for example.

[RESPONSE:] 


We assume that the reference to ``modern'' is a typo and that the
first sentence is meant to refer to traditional contraceptives.
To be clear, we argue that both modern and traditional contraceptive
use is intentional.
It is correct that we cannot directly rule out that some women will 
report abstinence as a contraceptive strategy even if the abstinence 
arise from something other than a desire to limit the risk of pregnancy.
It is important, however, to note that the question was specifically
about contraception.

\begin{quote}
Some couples use contraception methods to avoid pregnancy or to
space births. Are you currently using a method of contraception?
(For example, the pill, the IUD, condom, withdrawal, rhythm,
abstinence.)
\end{quote}


Although we cannot measure directly measure stress, we do have information 
on illnesses and their durations.
If stress is an important factor in reporting abstinence, we should
expect the same to be the case with illness, especially those that 
prevent a woman from working.
Hence, although we did not explicitly cover this possibility in the original 
version, coital frequency may also decrease because of illness.
We have added that possibility to the text.

% We would expect illness to go up both because of the direct effect
% on the household and from increased stress.

We show the effect of crop loss on illness in Table 5.
Crop loss is  not associated with substantial enough changes in BMI or 
self-reported illness for the women in the sample to explain our fertility results.
Furthermore, we ran regressions (not included in the paper) that show that there 
is only a very weak positive association between having been ill for two or more 
days in the last 6 months and the likelihood of reporting using abstinence 
as a contraceptive strategy.
This relationship becomes only marginally stronger, and never anywhere near
statistically significant, if defining illness as not being able to
perform usual activities for 7 days or more.
Going above 7 days makes the relationship negative, most likely because
very few women report being unable to perform regular activities for that
long.

To further investigate the role of abstinence we added a new
table to the Appendix, which further subdivides the results in Table 6.
Appendix Table A-8 shows, not surprisingly, that abstinence is the most 
effective traditional contraception, although only in combination 
with a crop loss as in the table in the main text.
A major caveat is that only 19 women report using abstinence only
across the three waves used for these regressions.
Modern contraceptive also are mainly effective in combination with
a crop loss, and the direct effect of using modern contraceptives
is not statistically significant.
We discuss these results toward the end of the section on 
``Conscious Decision or Unintended Consequences?''.



% Reduction in pregnancy is 0.3 for abstinence with crop loss
% Reduction in pregnancy is 0.2 for modern contraceptives with crop loss




\item The authors analyze the fertility of women who were married at the
start of the sample period. As a separate analysis, it would be
interesting to know whether marriage hazards change with crop loss in
the parents' home, although perhaps the authors do not have the sample
size to study this question.

[RESPONSE:] That is, indeed, a very interesting question, but we have 
decided not to pursue it at the moment for three reasons.
First, there are already two papers that examine this question, 
albeit for other countries \citep{mbiti08,Hoogeveen2011}.
Second, as those two papers show, the length of our paper would have
to increase substantially to do this question justice, and the Editor
has asked us to not increase the length of the paper.
Finally, the sample size is, indeed, on the small size to really get 
at this question.

We may return to this question in a separate paper.
A problem with the current literature on timing of marriage
is that it tend to focus on only the female side of the marriage market.
A more satisfactory model will have to consistently explain both sides 
of the market.
Hence, results that show, for example, that women are less likely to
marry after a crop loss would have to show the same for men. 


\item The survey measures crop loss continuously, and the indicator for
crop loss greater than 200 TZS is the authors' creation. They show in
the appendix that they obtain similar but less significant results if
they use the continuous measure (in levels or logs) or an indicator for
any crop loss. The 200 TZS seems arbitrary and possibly ex post, since
it leads to the most attractive results of the three sets of results
provided. It would be useful to more transparently show readers the
underlying variation and possible non-linearities, perhaps with a binned
version of the variable.

[RESPONSE:]

Our original discussion of this issue was scattered throughout the
original version of the paper.
We have now combined and expanded the discussion in the
Robustness section.
Before we respond, it is worth pointing out why we present the dummy
cut-off results in the paper.
We presented this paper a number of times before submission and whether 
we led with our dummy results or with the log results the discussant 
invariably asked us to focus on the other set of results.
We settled on the dummy results because they are easier to interpret 
for most people, but the downside is, as you point out, that the 
cut-off used is subject to discussion and that there is not one right level.

In many ways, the log specification in Appendix Table A-10 is more theoretically 
appealing because it captures that we should expect a given increase
in crop loss to have a diminishing impact on fertility related outcomes 
as the size of the crop loss increase (which is also why the linear version is 
not appealing; it assumes that a one TZS increase in crop loss has
the same effect whether it is at 0 or 50,000 TZS). 
Furthermore, the conclusion one would draw from the log results
is exactly the same as what we draw from the dummy approach.
Crop losses lead to statistically significant reductions in the likelihoods 
of pregnancy and birth, and statistically significant 
increases in the likelihood of using any contraceptives and using 
traditional contraceptives.
The differences from the dummy results are that the pregnancy and
birth results are slightly less statistically significant, whereas
the contraceptives results are slightly more statistically significant.
The main downside is that the log version is more difficult to interpret.

To show that the specific cut-off chosen does not have substantial effects on
the conclusions drawn, we have added Appendix Table A-9, which shows the main results
for different cut-offs together with the associated percentages of observations
that would be counted as having a crop loss.
Except for any crop loss (cut-off at zero TZS) the results
are consistent across all other specifications.
As the cut-off becomes very large statistically significance is lower
for pregnancy and births and eventually also for contraceptive use.
This is not surprising.
As we increase the cut-off there are more and more households 
that have experienced a crop loss, but are not counted as having a
crop loss, which will tend to bias results towards zero if there
is an effect of crop loss for levels lower than the cut-off 
(this is confirmed by regressions---not
shown---where we drop all observations with crop loss larger than zero but
smaller or equal to the cut-off and find substantially larger coefficients).




\item The authors use causal language throughout the paper, describing
their results as effects. Of course, crop loss may be far from randomly
assigned, so this language is not appropriate.

[RESPONSE:]

We have reduce the use of causal language throughout the paper and
added this as an additional caveat to our results in the Conclusion.
Below we discuss factors that may explain the distribution of
crop loss.

\item In the regression specifications, the authors express the survey
wave fixed effects as $D_{i}'\alpha$, where $D_{i}$ is a vector of survey
wave dummies. The subscript is incorrect, and it would be much more
natural to just write $\alpha_{t}$.

[RESPONSE:] We have changed the equation and the associated text.

\item What other time-varying covariates are available in the survey? It
would be interesting to consider the extent to which the crop loss
results are explained by other household-level idiosyncratic variation.

[RESPONSE:] 

There are three time-varying covariates in the survey that could 
potentially affect crop loss or the outcome variables directly:
rainfall, prices, and loss of livestock.
Other time-varying variables are illness and health variables, which 
we already show are unlikely to explain the results we find.

As we discuss in detail below in response to Reviewer 2, higher 
rainfall is associated with lower risk of crop loss, although there
are a number of caveats to that result.
Furthermore, including rainfall as an independent covariate does 
not change the relationship between crop loss and the five main outcomes.

In Appendix Table A-14 we show the main results when including a dummy for livestock
loss and the price index included with the Kagera data.
Although both new variables generally have the expected sign, neither are 
close to statistically significant and there is only minimal change in the 
estimated coefficients for crop loss.
There are two caveats to the results.
First, the loss of livestock question is not as ``clean'' as the
crop loss question in that the livestock question also includes 
any livestock given away by the household without specifying how much 
is actual loss and how much is gifts.
Second, the prices used for the price index are measured at the time
of the survey.
We tried lagging the price index, but that does not change the findings.

% - rainfall does not do anything
% - illness of mother and other household members 


\item The authors report some nonsensical coefficients and standard
errors like -0.000 (0.000). All coefficients and standard errors should
be written to two significant digits.

[RESPONSE:] The 0.000 coefficients were predominantly for variables
involving initial assets. 
We have changed the variable definition for assets, so that it is
now measured in 1,000,000 TZS. 

Note that there are still a few places---mainly for estimations
involving the use of modern contraceptives---where the estimated 
coefficients are so small that presenting additional digits does not make
much economic sense.
In those cases the standard errors are always equal to or larger than
the coefficients.
See, for example, Table A-10 on the effects of log crop loss.


\item Table 6 regresses pregnancy on contraceptive use, crop loss, and
their interaction. As the authors note, this regression is quite
difficult to interpret, given the endogeneity of contraceptive use. I
think this table adds little to the paper.

[RESPONSE:] Although we understand the concern about endogeneity, 
we believe the table is worth keeping. 
In the demography literature there is a substantial disagreement on whether
sustained fertility reduction requires access to modern contraceptives.
Part of the problem in this literature is that the ``intensity'' with
which a couple uses contraceptives---either traditional or modern---is
difficult to measure correctly.
% [something on the panel nature of the data and that that is important
% for getting at the incentives and the use over time; not perfect but
% at least is for a shorter period than other (cross-sectional) surveys]
The table is important because it provides us a way to get at this
intensity by showing that only couples that have a strong incentive to
prevent pregnancies appear to be successful users of contraceptives.
Furthermore, this shows that it is possible to postpone pregnancies with
just traditional contraceptives because traditional contraceptives account 
for the vast proportion of contraception used.

% [also speaks to whether this is a conscious decision or not] 
The table also supports our conjecture that the estimated effects 
% [might need a less causal word] 
of crop loss capture a conscious decision. 
If shocks lead to lower probability of pregnancy and birth because
of other factors than the decision to use contraceptive we should see
a negative and statistically significant coefficient for the crop loss
dummy by itself.
We do not.
Instead, what we get is a substantial negative coefficient for the interaction
between contraceptive use and crop loss and a coefficient for crop loss 
on its own that is positive and far from statistically significant.


\item Several appendix tables interact crop loss with initial assets in
Round 1 of the survey. This exercise is interesting and worthy of doing,
given that one might expect asset-poor households to have particular
difficulty in smoothing consumption. But the authors should leave Round
1 outcomes out of this regression.

[RESPONSE:] We have added the results where round 1 shocks has been 
excluded to the relevant table in the Appendix section on age
groups, wealth, and education.
The results are broadly consistent with what we originally found
for pregnancy and birth, 
The contraceptive use results have the correct sign, but the direct
effect is not statistically significant, presumably because of the 
lower variation in crop loss and the smaller sample size.


\end{enumerate}

% Overall, this is a simple but interesting paper that sheds light on
% important questions in the economics of fertility decisions before
% widespread adoption of modern contraception. It will make an incremental
% but solid contribution to the literature.


\newpage

\section*{Reviewer 2}

% This manuscript explores the effects of "unexpected" crop loss on
% fertility issues. The data come from four survey waves conducted in
% Tanzania, where there are repeated observations for the same individual.
% There are three main outcomes: births, pregnancies, and contraceptive
% use. The main estimation strategy relies on family fixed effects to
% address fixed differences (in wealth, education, etc.) that could be
% related to propensity to experience a crop loss. The crucial identifying
% assumption is that the timing of crop loss for a family is random and
% not related to other factors that might affect fertility.

The manuscript has a big positive: Fertility issues in developing
countries likely play an important role in economic growth, especially
via total completed fertility. Even if the shock in question did not
affect total completed fertility (i.e. individuals only postpone having
children), this paper would still be important since timing of births
matter. For example, being born in a period of low "resources" (food,
income, parental time) could irreparably harm a child's long-term
potential. To my knowledge, we know very little about the main
determinants of birth timing in a developing country context.

I have laid out some comments below in the interest of helping the
authors improve on a promising manuscript.


COMMENTS

\begin{enumerate}

\item Omitted variables

Even with family fixed effects, crop loss is potentially tied with
omitted variables. Crop loss is subjectively measured, so we'd need to
rely on the claim of the survey respondent that the loss was for
"uncontrollable" circumstances. The use of family fixed effects does
address fixed differences in wealth or education that might be related
to families at greater risk of crop loss. But, there could still be time
varying factors. We can't be certain that the crop loss is not the
result of some disease epidemic (or conflict) that also could affect
fertility.

Suggestion: The manuscript should be a little more forthcoming about
this potential problem. I would like to see an exploration into the
potential mechanisms behind the crop loss. Is it weather?

[RESPONSE:]

% [see response to R1's q7 on time-varying covariates]
% 
% [TK we need to tone down the causality part; explain the growing season and
% the crop loss question]

% The question ask (Question 11.B.9): ``Have you lost any part of the 
% harvested crop to insects, rodents, fire, rotting, etc.?''
It is correct that the crop loss is subjectively measured and that
it is not possible to verify what factors played a role in the crop loss.
We have therefore reduced the use of causal language as detailed in our
response to Reviewer 1 above.

As you suggest, weather is a potential factor.
Unfortunately the only available weather data consist of monthly rainfall 
at district level (there are five districts in the data) with missing 
information for some months.
Furthermore, there is no 30 year average to compare the observed 
district level rainfall data to.
This means that it is not possible to reliably separate out positive 
(above mean) and negative (below mean) rainfall shocks.
Although community or women fixed effects will partly ameliorate the
problem of missing average data, we cannot rule out the possibility
that all observed periods deviated substantially from the average in 
the same direction in which case we will fail to capture that the observed
rainfalls were all ``shocks''.

A number of different specifications of rainfall's association with 
crop loss is possible.
Kagera has two growing seasons, February through May and
October through December \citep[p 6]{Tanzania2007}.
We tried two different strategies: 
one where rainfall is measured as the average monthly rainfall 
over a set period prior to the survey month; 
and one based on average rainfall for growing season(s) preceding 
the survey month 
(that is, not including the current growing season if the survey
took place during a growing season).
Despite that rainfall is measured at district level either strategy
lead to variation across communities because of differences in 
when each community was surveyed.

The first strategy did not provide any clear results for a variety
of period lengths and we therefore focus on the growing season approach.
Appendix Table A-15 shows individual level fixed effects estimates of the 
effects of average monthly rainfall in centimeters on the likelihood
of experiencing a crop loss.
We use two specification, one with the last prior growing season and 
one with prior growing for waves 2 through 4 and the last two prior 
growing seasons for wave 1.
Higher rainfall is associated with a lower likelihood of a crop loss. 
A one standard deviation change in rainfall is associated with an
approximately 6 percentage point change in the probability of 
experiencing a crop loss.
We also tried adding square terms, but the estimated coefficients
were small and far from statistically significant.
The same is the case for using crop loss per capita as the dependent
variable. 

It is worthwhile noting the relatively small difference in the 
estimated effect of rainfall between the two models, 
although the coefficient on rainfall in Model I is just outside
the standard levels of statistically significance.
This suggests that the high level of crop loss observed in
wave 1 (see discussion below as well) is not mainly due to 
rainfall shocks that occurred longer ago than for the other waves.

In both models we include a dummy for missing information on rainfall 
for any month in the period / district combination.
If the rain data has one or more missing values for the months
covered in the growing season there is a large and statistically 
significant increase in the probability of reporting a crop loss
(dropping observations with missing information does little
to change the coefficients on rainfall, but do make them
statistically insignificant).
It is, however, very difficult to establish why this is because
no additional information in provided with the data.
Furthermore, the missing information problem only affects
observations in waves 2 and 3.

We also tried including rainfall---as defined above---as 
additional variables in the main regressions.
Consistent with the small effects of rainfall on crop loss
there are no apparent effects of the rainfall measures on the
outcomes and little to no change in the coefficients for crop
loss on our five main outcomes.

We realize that this answer does fully explain where the
crop losses come from but unfortunately there is not much
more relevant information that we can find beside the
rainfall data and the variables discussed in our response
to Reviewer 1.



% - problem is that we do not know when the crop loss occurred within
% the survey period (which is 12 months for the first round and therefore
% very long).





\item Trend

I am concerned by the fact that crop loss is much higher in the first
and second waves of the survey (see Table 1 on page 6). So, there's a
chance that the estimates are driven by some time trend that affects
family types that are more likely to experience crop losses. (The survey
wave fixed effects won't deal with this if there are differential trends
in the type of families who experience crop loss.)

Suggestion: Control for family characteristics (age, proxies for
education, proxies for wealth) at the first wave interacted with survey
wave. At the very least, this will help acknowledge the identification
threat for the reader.


[RESPONSE:] 

It is, indeed, potentially concerning that there are so substantial
differences in the proportion of households that experienced a 
shocks over the different rounds, although it is possible that
this simply is a stark reminder of the volatile nature of life
in a developing country.

We have added Appendix Table A-13 to address this
concern and discuss the results in the Robustness section. 
In addition to your suggestion of interacting a linear
time trend with wave 1 characteristics (assets per capita,
age group dummies, and education group dummies), we also present
a separate model where wave dummies are interacted with the
same set of wave 1 characteristics.
The small differences in point estimate and level of significance 
indicate that our results are not driven by differential trends
in which types of households that experience crop loss.


\item Spillovers

Neighbors may be affected an individuals crop loss. For one, price of
food goes up. This could be a positive income shock for families with
crops, but bad for families working in other areas.

Suggestion: Aggregate crop loss to the community level and use that as
the treatment variable.


[RESPONSE:]

We already include a version of this in the paper, but the 
results were not well explained.
Appendix Table A-16 include the fraction of households in the community 
who have experienced a crop loss (excluding the household itself).
The sign of the village level crop loss are generally the same as
those for the household crop loss, although---with the exception
of traditional contraceptives---the village level variable is not
statistically significant.
Furthermore, the crop loss coefficients decline only slightly
when including village level crop loss indicating that there is
only relatively mild spill-over between households for this
particular type of shock.

Finally, as detailed in our response to Reviewer 1 on potentially 
time-varying covariates we specifically discuss that prices do not appear 
to have any association with fertility decisions and including 
a price index does not the change the main results.

\item Births analysis

The births analysis automatically drops first wave of survey due to the
longer lag in reported crop loss (Table 3). However, I did not see this
mentioned anywhere.

[RESPONSE:]

The discussion of the longer lag was hidden in the middle of the 
Estimation Strategy section and the motivation was not clear.
We have rewritten that part of the Estimation Strategy to better
explain why the lag is there.

\end{enumerate}


\newpage
\bibliographystyle{econometrica}
\bibliography{../paper/incomeShocks-jde-r1}
\addcontentsline{toc}{section}{References}


\end{document}
