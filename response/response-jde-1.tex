\documentclass[letterpaper,12pt]{article}

% Xetex preamble
\usepackage{fontspec}
\setromanfont[Ligatures=TeX]{TeX Gyre Pagella}
\usepackage{unicode-math}
\setmathfont{TeX Gyre Pagella Math}

\usepackage[longnamesfirst]{natbib}
%\usepackage[tabhead,nolists,tablesfirst]{endfloat}
\usepackage[flushleft]{threeparttable}
\usepackage{booktabs}
\usepackage{rotating}  
\usepackage{amsmath}
\usepackage{caption} 
\usepackage{dcolumn} 
\usepackage{setspace}
%\usepackage{flafter} 
%\usepackage{longtable}
%\usepackage[pdftex]{graphicx}
\usepackage[xetex,colorlinks=true,linkcolor=black,citecolor=black]{hyperref} 
\usepackage[margin=1.0in]{geometry} 
\usepackage{multirow}

%opening
\title{} \author{}

\doublespacing

\begin{document}

\begin{center} \textbf{\large Income Shocks, Contraceptive Use, and
Timing of Fertility} \end{center}

\begin{center} Response to Referee Comments \end{center}

\noindent Please find attached the revised version of our paper,
``Income Shocks, Contraceptive Use, and Timing of Fertility.''
We would like to thank the two referees for the very useful comments and
suggestions,
which we believe has improved the paper.

[Additional discussion of overall things we have changed---if any]

We rewrote the code for cleaning and analyzing the data from the ground up 
to make it easier to follow and eventually publish on GitHub.
This rewrite slightly lowered the sample used by three observations.
Including all married/partnered women up to age 45, rather than exclude
women younger than 18, the final sample size is, however, the same as 
before and the results are essentially unchanged.

We list below our responses to the individual comments and suggestions.


\newpage

\section*{Reviewer 1}

% This interesting paper analyzes the relationship between income shocks
% (as proxied by self-reported crop loss) and marital fertility, using a
% small Tanzanian panel survey from the early 1990s. A key feature of the
% paper is that the authors are able to observe use of both traditional
% and modern birth control techniques. In regression specifications with
% and without community or woman fixed effects, the authors find that crop
% loss is associated with decreases in the probability of conceiving, with
% a corresponding increase in use of traditional birth control techniques
% but not modern contraceptives.

The paper could benefit from focusing its motivation somewhat more. An
important point (that the authors do not emphasize enough) is that use
of modern contraception in the sample is extremely low, so the lack of a
relationship between shocks and use of modern methods is not surprising.
The more noteworthy result is that use of traditional methods responds
to shocks. In this sense, the paper makes an important contribution to
understanding how populations control fertility before widespread
adoption of modern contraception. For instance, it is well known that
marital fertility in Europe declined largely before modern contraception
became available, but the methods couples used to achieve this reduction
are not well documented (see, for example, Guinnane JEL 2011). The paper
would be even more compelling if the authors focused the paper's
motivation on the questions of whether and how couples modify fertility
to respond to changing economic conditions before the advent of
modern contraception. That motivation would also help justify their
choice of dataset, which otherwise seems somewhat small and out of date
relative to other available panel datasets from developing countries.

[RESPONSE:]

Specific comments:

\begin{description}

\item The authors interpret their results as postponement, implying they
represent a change in the timing of fertility but not its lifetime
cumulation. That is likely to be the correct interpretation, but can
they really tell? It seems to me that the current fixed effect model
does not tell us. Perhaps they could include 1-2 lags in the crop shock
variable to shed some light on it.

[RESPONSE:]

\item The authors argue that the increase in modern contraceptive use is
intentional, but the evidence for this interpretation is weaker than
they would like. As they do note, while the survey asks about using
abstinence specifically to prevent births, there is much scope for
misreporting, and many women may claim that reductions in coital
frequency were related to fertility planning even if they were not. As
their main evidence against this concern, the authors show that crop
loss has small effects on labor supply. But this finding is only
relevant for the very specific theory that all unintentional abstinence
is related to a work-sex tradeoff. Couples may have less sex during hard
times because they are stressed, for example.

[RESPONSE:]

\item The authors analyze the fertility of women who were married at the
start of the sample period. As a separate analysis, it would be
interesting to know whether marriage hazards change with crop loss in
the parents' home, although perhaps the authors do not have the sample
size to study this question.

[RESPONSE:]

\item The survey measures crop loss continuously, and the indicator for
crop loss greater than 200 TZS is the authors' creation. They show in
the appendix that they obtain similar but less significant results if
they use the continuous measure (in levels or logs) or an indicator for
any crop loss. The 200 TZS seems arbitrary and possibly ex post, since
it leads to the most attractive results of the three sets of results
provided. It would be useful to more transparently show readers the
underlying variation and possible non-linearities, perhaps with a binned
version of the variable.

[RESPONSE:]

\item The authors use causal language throughout the paper, describing
their results as effects. Of course, crop loss may be far from randomly
assigned, so this language is not appropriate.

[RESPONSE:]

\item In the regression specifications, the authors express the survey
wave fixed effects as $D_{i}'\alpha$, where $D_{i}$ is a vector of survey
wave dummies. The subscript is incorrect, and it would be much more
natural to just write $\alpha_{t}$.

[RESPONSE:] We have changed the equation and the associated text.

\item What other time-varying covariates are available in the survey? It
would be interesting to consider the extent to which the crop loss
results are explained by other household-level idiosyncratic variation.

[RESPONSE:]

\item The authors report some nonsensical coefficients and standard
errors like -0.000 (0.000). All coefficients and standard errors should
be written to two significant digits.

[RESPONSE:] The 0.000 coefficients were predominantly for variables
involving initial assets. 
We have changed the variable definition for assets, so that it is
now measured in 1,000,000 TZS. 

Note that there are still a few places---mainly for estimations
involving the use of modern contraceptives---where the estimated 
coefficients are so small that presenting additional digit does not make
much economic or practical sense.
In those cases the standard errors are always equal to or larger than
the coefficients.
See, for example, Table A-8 on the effects of log crop loss.


\item Table 6 regresses pregnancy on contraceptive use, crop loss, and
their interaction. As the authors note, this regression is quite
difficult to interpret, given the endogeneity of contraceptive use. I
think this table adds little to the paper.

[RESPONSE:]

\item Several appendix tables interact crop loss with initial assets in
Round 1 of the survey. This exercise is interesting and worthy of doing,
given that one might expect asset-poor households to have particular
difficulty in smoothing consumption. But the authors should leave Round
1 outcomes out of this regression.

[RESPONSE:]


\end{description}

% Overall, this is a simple but interesting paper that sheds light on
% important questions in the economics of fertility decisions before
% widespread adoption of modern contraception. It will make an incremental
% but solid contribution to the literature.


\newpage

\section*{Reviewer 2}

% This manuscript explores the effects of "unexpected" crop loss on
% fertility issues. The data come from four survey waves conducted in
% Tanzania, where there are repeated observations for the same individual.
% There are three main outcomes: births, pregnancies, and contraceptive
% use. The main estimation strategy relies on family fixed effects to
% address fixed differences (in wealth, education, etc.) that could be
% related to propensity to experience a crop loss. The crucial identifying
% assumption is that the timing of crop loss for a family is random and
% not related to other factors that might affect fertility.

The manuscript has a big positive: Fertility issues in developing
countries likely play an important role in economic growth, especially
via total completed fertility. Even if the shock in question did not
affect total completed fertility (i.e. individuals only postpone having
children), this paper would still be important since timing of births
matter. For example, being born in a period of low "resources" (food,
income, parental time) could irreparably harm a child's long-term
potential. To my knowledge, we know very little about the main
determinants of birth timing in a developing country context.

I have laid out some comments below in the interest of helping the
authors improve on a promising manuscript.


COMMENTS

\begin{description}

\item Omitted variables

Even with family fixed effects, crop loss is potentially tied with
omitted variables. Crop loss is subjectively measured, so we'd need to
rely on the claim of the survey respondent that the loss was for
"uncontrollable" circumstances. The use of family fixed effects does
address fixed differences in wealth or education that might be related
to families at greater risk of crop loss. But, there could still be time
varying factors. We can't be certain that the crop loss is not the
result of some disease epidemic (or conflict) that also could affect
fertility.

Suggestion: The manuscript should be a little more forthcoming about
this potential problem. I would like to see an exploration into the
potential mechanisms behind the crop loss. Is it weather?

[RESPONSE:]


\item Trend

I am concerned by the fact that crop loss is much higher in the first
and second waves of the survey (see Table 1 on page 6). So, there's a
chance that the estimates are driven by some time trend that affects
family types that are more likely to experience crop losses. (The survey
wave fixed effects won't deal with this if there are differential trends
in the type of families who experience crop loss.)

Suggestion: Control for family characteristics (age, proxies for
education, proxies for wealth) at the first wave interacted with survey
wave. At the very least, this will help acknowledge the identification
threat for the reader.


[RESPONSE:]


\item Spillovers

Neighbors may be affected an individuals crop loss. For one, price of
food goes up. This could be a positive income shock for families with
crops, but bad for families working in other areas.

Suggestion: Aggregate crop loss to the community level and use that as
the treatment variable.


[RESPONSE:]


\item Births analysis

The births analysis automatically drops first wave of survey due to the
longer lag in reported crop loss (Table 3). However, I did not see this
mentioned anywhere.

[RESPONSE:]


\end{description}


\newpage
\bibliographystyle{econometrica}
\bibliography{../paper/incomeShocks-jde-r1}
\addcontentsline{toc}{section}{References}


\end{document}
